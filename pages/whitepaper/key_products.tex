\section{Key products}

% Products
Products of the precision medicine unit include
\acmguru for variant classification and interpretation,
\deepinfer for defining the posterior probability of every genetic determinant of disease based on all known public data,
\archipelago for statistical interpretation of variant set association testing (VSAT),
\skatrbrain for automated statistical genomics,
\macat for multi-omic joint analysis of VSAT by aggregated Cauchy association test (ACAT),
\dnasnake for DNA pre-processing
\rnasnake for RNA pre-processing, 
and the documentation of pipelines in
\pipedevdocdna. 

\subsection{Product example: advanced DNA sequencing data preprocessing}

\dnasnake represents a vital product from our precision medicine unit, meticulously designed to preprocess WGS DNA data for use in clinical genetics reporting, statistical analysis, and machine learning applications. 
Employing the Genome Analysis Toolkit (GATK), \dnasnake standardises the preparation of DNA sequencing data to ensure consistency and reliability across diverse analytical applications.

%\subsubsection{overview}

\dnasnake is engineered around the GATK best practices for DNA sequence data preprocessing. This workflow is integral to producing high-quality, clinical-grade DNA data outputs, which are crucial for downstream processes like variant interpretation in ACMGuru and assessing genetic determinants of disease with deepInfeR.

%\subsubsection{workflow description}

The \dnasnake workflow is detailed and robust, encompassing several critical stages of DNA preprocessing:

\begin{enumerate}
    \item \textbf{Quality Control and Pre-processing}: Initial receipt of raw FASTQ files followed by quality control assessments using tools like FastQC and subsequent trimming with Trimmomatic.
    \item \textbf{Alignment}: Alignment of sequences to the GRCh38 reference genome using the Burrows-Wheeler Aligner (BWA).
    \item \textbf{Post-alignment Optimization}: Includes marking duplicates with Picard Tools, and base quality score recalibration (BQSR) with GATK’s BaseRecalibrator and ApplyBQSR tools.
    \item \textbf{Variant Calling}: Utilizing GATK’s HaplotypeCaller for calling germline SNPs and indels, followed by variant quality score recalibration (VQSR) to ensure high-quality variant calls.
    \item \textbf{Output Generation}: Production of annotated, processed BAM and VCF files ready for comprehensive genetic analysis.
\end{enumerate}

%\subsubsection{integration with other systems}

Processed outputs from DNAsnake feed directly into:
\begin{itemize}
    \item \acmguru{}, for detailed variant classification and interpretation in line with ACMG guidelines.
    \item \deepinfer{}, which utilizes the processed data to calculate the posterior probabilities of genetic variants, influencing disease phenotypes based on extensive public data repositories.
\end{itemize}

%\subsubsection{key benefits}

By automating the WGS DNA data preprocessing with \dnasnake, our unit will achieve:
\begin{itemize}
    \item \textbf{Standardisation and reproducibility}: Ensures that all samples are processed through a uniform pipeline, reducing variability and enhancing the reliability of results.
    \item \textbf{Efficiency and scalability}: Capable of handling large-scale datasets with the flexibility to accommodate increasing data volumes without sacrificing performance.
    \item \textbf{Integration and Interoperability}: Seamlessly interfaces with other analytical tools and databases, promoting a cohesive and integrated approach to precision medicine.
\end{itemize}

%\subsubsection{conclusion}

\dnasnake exemplifies our commitment to delivering state-of-the-art solutions for genetic data preprocessing. As an element of our single-source management strategy, it not only supports but enhances the capabilities of our precision medicine initiatives, ensuring that data used across various platforms is of the highest quality and utility.
