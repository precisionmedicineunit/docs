\section{Standards and accreditation}

As we progress, our data management practices will evolve towards adherence to recognised standards, aiming to ensure robustness data handling. 

Our omics data processing involves rigorous analysis and annotation methodologies. For example, we interpret genetic determinants of disease using a comprehensive evidence-based approach, adhering to internationally recognized standards, including:

\begin{itemize}
  \item Joint consensus recommendations for the interpretation of sequence variants by the American College of Medical Genetics and Genomics and the Association for Molecular Pathology.
  \item Guidelines for reporting sequence variants in cancer developed collaboratively by the Association for Molecular Pathology, American Society of Clinical Oncology, and College of American Pathologists.
  \item Technical standards for reporting constitutional copy-number variants, as recommended by the American College of Medical Genetics and Genomics and the Clinical Genome Resource.
  \item Additional methodologies for variant filtering, interpretation, and prioritisation in genetic studies.
\end{itemize}
These guidelines ensure that our analytical processes are aligned with recognised standards of accuracy and clinical relevance, to deliver precise and actionable genetic insights
\citep{richards2015standards,
li2017standards,
riggs2020technical,
li2017standards,
pedersen2021effective,
li2017intervar,
xavier2019tapes}.

As our projects expand, we can work towards implementing standards such as \href{https://www.iso.org/standard/67888.html}{ISO 23092}, which concerns genomic information representation, to enhance data integrity and accessibility.
We also aim to align with ISO 23092-4 and ISO 23092 standards for the reference software and transport and storage of genomic information, respectively, to support our framework for handling sensitive genetic data securely and efficiently. 
Our facilities will grow with compliance in mind, such as \href{https://www.iso.org/standard/56115.html}{ISO 15189}, a critical standard for medical laboratories that ensures the quality and consistency of clinical genetic analysis. 
This standard is in place for Geneva health 2030 genome center, which provides us with clinical-grade sequencing \href{https://www.health2030genome.ch/dna-sequencing-platform/}{(Geneva Health 2030)}.
\begin{itemize}
	\item Genomic information representation - \href{https://www.iso.org/standard/67888.html}{ISO 23092}
    \item Reference Software - \href{https://www.iso.org/standard/75859.html}{ISO 23092-4}
    \item Transport and Storage - \href{https://www.iso.org/standard/79882.html}{ISO 23092}
    \item Medical laboratory quality and competence - \href{https://www.iso.org/standard/56115.html}{ISO 15189}
\end{itemize}

\subsection*{Ethical and legal framework}
Guidance on ethical and legal considerations is sourced from the Global Alliance for Genomics and Health (GA4GH), which provides comprehensive resources on the regulatory and ethics toolkit for genomics research \href{https://www.ga4gh.org/genomic-data-toolkit/regulatory-ethics-toolkit/}{(GA4GH Toolkit)}. 
Swizterlan-based companies
\href{https://www.sophiagenetics.com/}{SophiaGenetics} and \href{https://varsome.com}{VarSome},
and US-based
\href{https://blueprintgenetics.com/certifications/}{(BlueprintGenetics) 
serve as a model for implementing these standards, noted for their rigorous certification process in the same areas as the precision medicine unit.

\section{Compliance with Swiss Law}

Our work with omic data at the hospital will adhere to Swiss legal provisions on genetic testing. 
Under the oversight of the Federal Council, specifically the Federal Department of Home Affairs (FDHI) and the Federal Office of Public Health (FOPH), we ensure all genetic tests, including those for pharmacogenetics, comply with the Federal Act on Human Genetic Testing (HGTA).

Relevant regulations are detailed in:
\begin{itemize}
  \item \href{https://www.fedlex.admin.ch/en/cc/internal-law/810.1}{Federal Act of 8 October 2004 on Human Genetic Testing (HGTA)}.
  \item \href{https://www.fedlex.admin.ch/en/cc/internal-law/810.122.1}{Ordinance on Human Genetic Analysis (OAGH)} from February 14, 2007.
  \item \href{https://www.fedlex.admin.ch/en/cc/internal-law/810.122.2}{Ordinance on DNA Profiling in Civil and Administrative Matters (OACA)} also from February 14, 2007.
\end{itemize}

These laws and ordinances govern the conduct of genetic analysis, ensuring that all procedures respect the principles of medical and human dignity. They also regulate the use of genetic data for medical, civil, and administrative purposes, supporting the ethical use of genetic tests to influence treatment plans and improve patient care. Further details can be accessed through \href{https://www.fedlex.admin.ch/fr/cc/internal-law/81#810.12}{Fedlex}, which outlines the full scope of genetic legislation relevant to our operations.

